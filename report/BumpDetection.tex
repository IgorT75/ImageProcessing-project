\documentclass[14pt, letterpaper]{article}
\makeatletter
\def\sectionsuffix{.}

\renewcommand\@seccntformat[1]{\csname the#1\endcsname%
	\csname#1suffix\endcsname\quad}
\makeatother

%opening
\title{Bump detection algorithm - due on 9/22}
\author{Igor Trubnikov}
%\date{}

\usepackage[utf8]{inputenc}
%\usepackage[english]{babel}
\usepackage{braket}
\usepackage[margin=0.5in]{geometry}
\usepackage{amsmath, amsfonts, amssymb, amsthm}
\usepackage{bbold}
\usepackage{graphicx}
\usepackage{fullpage}
\usepackage{scrextend}
\usepackage{hyperref}

\newcommand{\bd}{\textbf}
\usepackage[framed,numbered,autolinebreaks,useliterate]{mcode}

\hypersetup{
	colorlinks=true,
	linkcolor=blue,
	filecolor=magenta,      
	urlcolor=cyan,
	pdftitle={Sharelatex Example},
	bookmarks=true,
	%pdfpagemode=FullScreen,
}

\begin{document}

\maketitle
\tableofcontents
\newpage

\section{Introduction}
Write a Matlab code to generate a zig-zag sequence of DCT coefficients and apply it to slide\#25
of Lecture\#4 to generate the following sequence in slide\#34.\\
-26* -3 1 -3 -2 -6 2 -4 1 -4 1 1 5 0 2 0 0 -1 2 0 0 0 0 0 -1 -1 EOB\\
(: DC in the prev block is assumed -17 for DPCM)\\\\
\bd{Solution:}\\Please find Matlab code below.\\
It's also attached - see file \bd{Problem1.m}.
%\lstinputlisting[language=MATLAB]{../Problem1/Problem1.m}

\newpage

\section{Description of the algorithm}
Debug the Matlab code below for Huffman coding of DC coefficients and show the working code
and indicate where you made modification(s)
%\lstinputlisting[language=MATLAB]{../Problem2/Problem2_original.m}
\bd{Solution:}
\\Found following issues in the code:
\begin{enumerate}
	\item There is a mistake in the definition of table matrix - table below shows that value in the 1st column in 6th row should be 3, not 5.\\
	\item I also had to add \mcode{endif} to the definition of \mcode{jpg\_dc\_enc} function.\\
	As far as I understand it's supposed to return category number from the Table \ref{tab:categories}.\\
	But somehow there are 2 branches of \mcode{if} one of which returns \mcode{[0 1 0]} (which is a base code for \mcode{0} while another assigns \mcode{floor(log2(abs(x)))+1} to \mcode{c} variable.\\\\I assumed that it's supposed to return category number - see the corrected code.\\
	\begin{table}[!htbp]
		\centering
		%\includegraphics[totalheight=8cm]{./images/JPEGCoefCodingCategories.jpg}
		\caption{JPEG coefficient coding categories.}
		\label{tab:categories}
	\end{table}
	\item It looks like Octave does not have \mcode{int2bin} function implemented. I just assigned \mcode{tmp} to \mcode{[0 1 1 1]} in code manually to make it work.
	\item Book says in example that -9 is represented as 0111 or rather \mcode{[0 1 1 1]} in MATLAB. So if leftmost bit is 0 then it's a negative number. Original code from the problem assumes opposite.
	\item Book also says that: "For a general DC difference category (say, category $K$), an additional bits are needed and computed as either
	the LSBs of the positive difference or the LSBs of the negative difference minus 1."\\
	This means we should subtract \mcode{ones(1, c)} from \mcode{tmp(2:c+1)}. Original example assumes the opposite.
	\item I also introduced \mcode{categoryToLength} array to encode length of each category - see Table \ref{tab:defaultDCcode}.\\
	It's used to compute every bit of the LSBs value. Variable \mcode{LBSLength} represents length of the LBS bits sequence.
	\begin{table}[!htbp]
		\centering
		%\includegraphics[totalheight=4cm]{./images/JpegDCEndTable.jpg}
		\caption{JPEG default DC code (luminance).}
		\label{tab:defaultDCcode}
	\end{table}
\end{enumerate}

\section{Verification}
All the changes described above are made - although there is a hardcode for particular number because I don't have \mcode{int2bin} function.\\
Please find MATLAB code below. It's also attached to the homework - see file \bd{Problem2.m}.


\section{Conclusion}
Check it out

\end{document}
